\chapter{Conclusion and Future Work}
In this chapter, the conclusions derived from the implementation and evaluation of the In-Network Event Processing framework are presented. The future work that can be undertaken to build up on the implementation and the conceivable improvements are additionally discussed. 




\section{Conclusion}
Network virtualization and software-defined networking offer boundless possibilities for provisioning chained network functions on demand with the aid of software-based solutions and programmable network control planes. As part of research conducted in the thesis, an exercise in programming the network control plane with application context and enabling the data plane to process application logic is presented within the context of a complex event processing ecosystem. To achieve the goals of the research the following contributions have been made:
\begin{itemize}
	\item An event processing framework is implemented within the highly adopted Open vSwitch.
	\item The vSwitch is enabled to perform logical and stateful operations based on user logic configured as event rules.
	\item A framework to remotely offload event rules onto the network control plane via HTTP is implemented using the RYU controller.
	\item A thorough evaluation of the implementation against several parameters is detailed and discussed.
\end{itemize}
The results of the evaluation demonstrate that the benefits of detecting and redirecting events at the vSwitch are compelling. In this model, the vSwitch assumes the role of an in-network broker. Evaluation of this model shows a reduction in point-to-point latency between the producers and consumers of events. By avoiding the utilization of and context switch to a broker application, a significant reduction in network traffic and processing is achieved for single staged systems. These results when extrapolated to multi-staged processing systems can potentially avoid multiple context switches and thereby improve the latency significantly, and reduce the burden on the network. However, the results also show a modest increase in the number of processing cycles per packet; which when weighed against the added benefits is a reasonable price to pay. \newline \newline When higher level logical and stateful operations are performed on event attributes, the benefits are less apparent in the current implementation. Although a reduction in network traffic, prevention of context switch to a broker and consequent avoidance of broker processing are observed, the number of processing cycles required per packet increases significantly because of reliance on the OpenFlow processing pipeline instead of the cache. In addition to the impact on the event processing pipeline, this also adversely impacts the generic performance of the Open vSwitch. A possible solution to this problem is presented in Section 6.1. Overall, the thesis elaborates on the potential of offloading aspects of event processing onto the underlying network. Although stateful event operations are implemented, the benefits are not apparent because of the current caching limitations. Nonetheless, while performing the role of an event broker, the benefits become more apparent. This provides network operators with promising avenues to explore models of complementing existing complex event processing ecosystems with highly tuned application-aware custom network solutions.

\section{Future Work}
In the current implementation, event processing actions do not take advantage of the megaflow cache of the Open vSwitch. Instead, each event has to be looked up the OpenFlow processing pipeline and event actions have to be applied for accurate results. To achieve the same, the megaflow cache eviction rate is increased which hits the performance of the Open vSwitch bridge, affecting all the systems bridged using Open vSwitch and not just the implemented event processing pipeline. To avoid this problem, a sophisticated re-validator thread may be developed to evict only event based rules from the cache and allow other rules to remain cached. \newline \newline
The implemented event processing within Open vSwitch results in a significant increase in processing cycles per packet. This is because, for each packet, the application layer is accessed, event attributes are extracted and deserialized for further processing. This adds significant cycles per packet. Future work may address this drawback to make the event extraction process much leaner than what it currently is.\newline \newline
The event operations support in the current implementation is limited. With the main goal of the thesis - implementing an event processing framework - achieved, future works may now extend the framework to add support for many other operations, stateful or logical.\newline \newline
The current implementation focuses on the UDP transport protocol. Future works may extend the support to other network protocols. Furthermore, The system model used in the current implementation considers event type and two event attributes of integer event types. Future works may focus on extending the system model to support more attributes with floating point types.  

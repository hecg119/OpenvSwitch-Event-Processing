\chapter{Discussion} \label{chap:Discussion}
Complex Event Processing (CEP) ``is a defined set of tools and techniques for analyzing and controlling the complex series of interrelated events that drive modern distributed information systems.`` \citep{Luckham2008}. The Event Processing paradigm sees the world as a series of discrete events. The producers and consumers of the events are spatio-temporally decoupled. The produced events are distributed to the consumers through a notification service, which may be a dedicated channel or distributed brokers. The following describes and expresses a complex event processing abstraction on an Openvswitch. The implementation entails extension of the OpenvSwitch 2.6.1 to support OpenFlow rules for events and a northbound API to install event rules using a ryu controller.



\chapter{Requirements}

\section{\textbf{Event Semantics}}
The system model treats the flow of packets at a physical port as an event sink. The event data, encapsulated within a packet, is extracted and mapped to a defined minimal Event representation. Each event follows the below representation:\\1. Unique Identifier\\
2. Event Type\\3. Event Attributes 

Let 'E' be an event of type 't' and attributes '$a_1$','$a_2$'..'$a_n$'.

\begin{equation}
E\lbrace t,a_1,a_2..a_n \rbrace
\end{equation}

\begin{flushleft}
Event instances can be detected based on different variants.
\\1. Detect based on Event Type
\\2. Detect based on values in Event Attributes
\\3. Detect based on number of instances within a window.
\\
\end{flushleft}
\begin{flushleft}
To this end, an event algebra is realized within the virtual switch:
\\
\textbf{Comparator}: Event is signalled when comparator operator evaluates to True. Comparators are used to compare Event Type and Attributes with a value. Comparator operations can be used in combination with other operations to detect complex event patterns.
\\ 
\textbf{Conjunction}: Event is signalled when all constituent input patterns evaluate to True. The constituent input patterns are composed using the comparator operations.
\\
\textbf{Disjunction}: Event is signalled when atleast one constituent input pattern evaluates to True. The constituent input patterns are composed using the comparator operations.
\\
textbf{Sequence}: Event is signalled when input patterns are detected in a defined sequence.
\\
\textbf{Window}: Event is signalled when input patterns are detected within a time window.
\end{flushleft}

\begin{flushleft}
A set of comparator operators is given by:
\begin{equation}
\oplus  \ni  \lbrace =,<,>, \rbrace
\end{equation}
the event operations using the comparator operators are expressed as below in (2.3):
\begin{equation}
E.a \oplus intval
\end{equation}
\end{flushleft}

\begin{flushleft}
Using the comparator operator and the previously defined Event Algebra, other complex operations are expressed:
\begin{equation}
E.t = strval \wedge (E.a_1 \oplus intval_1 \wedge E.a_2 \oplus intval_2)
\end{equation}
\begin{equation}
E.t = strval \vee (E.a_1 \oplus intval_1 \wedge E.a_2 \oplus intval_2)
\end{equation}
\begin{equation}
E.t = strval \vee (E.a_1 \oplus intval_1 \vee E.a_2 \oplus intval_2)
\end{equation}
\begin{equation}
(E.a \oplus intval) Sequence (E.a \oplus intval)
\end{equation}
\begin{equation}
(E.a \oplus intval) \wedge (E.a \oplus intval) \wedge Window = intval
\end{equation}
\end{flushleft}

\begin{flushleft}
The expressions are installed in the switch in the form of flow rules. A mechanism is provided to install the flow rules via JSON API through a controller.

Figure to be attached.
\end{flushleft}









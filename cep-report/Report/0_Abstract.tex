\begin{abstract}
	In today's ubiquitous computing environment, the demands to filter data from the noise and react to patterns of events expeditiously has intensified the burden put on the already complex event processing ecosystem. The complex event processing ecosystem is composed of several contributing technologies working in synergy to afford the needed processing for computation and communication as events progress through the system achieving higher levels of abstractions with producers and consumers at each level. The ecosystem can be viewed as an elaborate overlay on top of the underlying network. The thesis proposes a paradigm of viewing the underlying network as a contributing technology to the complex event processing ecosystem by offloading event processing application context onto the network. With the advent of software defined networking and its consequent separation of the control and data planes, the possibilities to tune network services to the bidding of user applications are immense. Whereas network function virtualization aspires to virtualize diversified network hardware into effortlessly serviced software solutions provisioned on commodity servers, the thesis aims to research the implications of moving event processing application context onto virtualized network components. The hypothesis, to begin with, is that offloading of application context onto the underlying network allows network architects to tune their services to latency sensitive applications. \newline\newline
	To achieve the goal an event processing framework is implemented within a highly adopted multi-server production virtual switch. The implemented framework equips the vSwitch with capabilities to detect and process events based on configured event rules. Additionally, an \ac{sdn} controller based northbound \ac{api} is implemented to offload event rules onto the vSwitch remotely.
	\newline \newline
	The results of the evaluation demonstrate that the benefits of detecting and redirecting events at the vSwitch are compelling. In our evaluation of a single stage processing system, with the vSwitch performing the role of an in-network event broker, a reduction in point-to-point latency between the producers and the consumers of the events is observed along with a scale down in network traffic and complete avoidance of application broker processing. When logical and stateful actions are applied on event attributes, with the vSwitch performing the role of an in-network event processor, the latency benefits are balanced out because of the adopted caching policy  which results in a significant increase in processing per packet. However, a scale down in network traffic and avoidance of application broker processing is achieved. \newline
	Overall, the thesis elaborates on the potential of offloading aspects of event processing onto the underlying network. The observed results provide network operators with promising avenues to explore models of complementing existing complex event processing ecosystems with highly tuned application-aware custom network solutions.
\end{abstract}

